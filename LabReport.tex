%======   Template created by Jonathan Blair  ========
%=====================================================

%=====================================================
%============ Controls ===============================
%=====================================================

\documentclass[12pt,letterpaper,onecolumn]{article}
%\documentclass[11pt,letterpaper,onecolumn]{article}
%\documentclass[10pt,letterpaper,onecolumn]{article}
%\documentclass[12pt,letterpaper,twocolumn]{article}
%\documentclass[11pt,letterpaper,twocolumn]{article}
%\documentclass[10pt,letterpaper,twocolumn]{article}


\usepackage{amsmath}
\usepackage{graphics}
\usepackage{graphicx} %more modern version of graphics
%\graphicspath{{path-to-folder-containing-necessary-graphics}{other folder as necessary}}


%=====================================================
%============ \begin{document} =======================
%=====================================================

\begin{document}

%=====================================================
%============ Title ==================================
%=====================================================

\title{Investigating Temperature and Voltage Dependence in NPN Bipolar Junction Transistors}
%\title{\Large\bf Larger, Bolded Title}

%=====================================================
%============ Author =================================
%=====================================================
\author{
 Garik Livingston and Lily Nguyen \\*
  \\*
 PHY 353L Modern Laboratory \\*
 Department of Physics \\*
 The University of Texas at Austin \\*
 Austin, TX 78712, USA
}
\date{February 11, 2025}

%\address{The University of Texas, Austin, Texas, 78712}

\maketitle

%=====================================================
%============ Abstract ===============================
%=====================================================

\begin{abstract}
	In this experiment, we varied the voltage supplied to the base of a npn BJT transistor and measured the current running through the collector to form a data set which we fitted to the Ebbers-Moll equation.
	From this fit we calculated the thermal voltage ($V_T$) and saturation current ($I_s$). Both these paramaters have strong temperature dependence and from our fit we determined an experimental value for the boltzman constant of $k = 1.378 \times 10^{-23} \pm 8.4 \times 10^{-26} \frac{J}{K}$ for the silicon transistor and a value of k = $5.6 \times 10^{-23}\pm 4.000 \times 10^{-24} \frac{J}{K}$ for the germanium transistor. 
	We also determined an experimental value for the band gap of the transistors with a $E_g = 1.48 \pm 0.012eV$ for silicon and  $E_g = 0.158 \pm 0.003 eV$ for germanium.
	
\end{abstract}

%=====================================================
%============ Body of the article ====================
%=====================================================

%=====================================================
%============ Section ================================

\section{Introduction}

\subsection{Physics Motivation}

Semiconductors are essential components of modern electronics due to their unique electronic properties: a valence band filled with electrons and a nearly empty conduction band, separated by an energy band gap \( V_g \). The ability of electrons to transition from the valence band to the conduction band under specific conditions, such as thermal excitation or applied voltage, forms the basis for the functionality of semiconductor devices like transistors \cite{Thornton}.

In this experiment, we investigate the behavior of an NPN bipolar junction transistor (BJT) by examining the relationship between the collector current \( I_C \) and the base-emitter voltage \( V_{BE} \). The transistor’s function depends on the exponential dependence of \( I_C \) on \( V_{BE} \), described by the Ebers-Moll equation:

\[
I_C = I_S \left(e^{eV_{BE}/kT} - 1\right)
\]

where \( I_S \) is the saturation current, \( e \) is the elementary charge, \( k \) is Boltzmann’s constant, and \( T \) is the absolute temperature \cite{Neudeck}. This exponential behavior arises from the Boltzmann factor, which governs the probability of electron excitation across the band gap:

\[
P(E) \propto e^{-E/kT}
\]

By measuring \( I_C \) as a function of \( V_{BE} \) at different temperatures, we can extract both Boltzmann’s constant \( k \) and the band gap \( V_g \) of the semiconductor material. These measurements not only reinforce the theoretical principles of semiconductor physics but also demonstrate the practical application of transistors in amplification and switching, which are fundamental to all modern electronic devices \cite{Thornton}.	
%Broad physics motivation should be discussed briefly but
%meaningfully. Basic phenomena should be
%explained (or referred to) and
%prediction for experimental results clearly
%stated. Here and throughout the report appropriate
%references should be included~\cite{book, article}.

\subsection{Historical context}

%You may want to relate what you are doing to first or previous
%work on this topic. Since you are doing an experimental work,
%the context should be on the experimental technique. For example,
%you may say that this was first done in a such and such way
%but later it was discovered
%that one can also do it another way. Your technique may be
%related to the first or none of the above.

The invention of the bipolar junction transistor (BJT) in 1947 by John Bardeen, Walter Brattain, and William Shockley marked a pivotal moment in semiconductor technology. Their initial device, the germanium point-contact transistor, exhibited power gain but was unstable and difficult to reproduce due to sensitivity to surface states and impurities. Shockley later developed the p-n junction transistor, which proved more reliable by utilizing bulk conduction of minority carriers. This advancement earned the trio the Nobel Prize in Physics in 1956 \cite{Lukasiak}.

Early transistors were difficult to manufacture consistently. The development of grown junction transistors in 1952 improved stability, but the process remained complex and wasteful. The introduction of the alloyed junction transistor in the same year simplified production and reduced material waste. By 1954, diffused junction transistors allowed for more precise control over transistor properties, enabling higher-frequency operation. The shift from germanium to silicon further improved performance due to silicon's lower reverse currents and better thermal stability. The first commercial silicon transistors were produced by Gordon Teal in 1954, with planar transistors introduced by Jean Hoerni in 1960, enabling mass production and miniaturization \cite{Lukasiak}.

A crucial distinction between germanium and silicon lies in their respective band gaps. Germanium, with a smaller band gap of approximately \(0.67\,eV\), exhibits higher intrinsic carrier concentrations and is more sensitive to temperature changes. In contrast, silicon, with a larger band gap of \(1.1\,eV\), offers better thermal stability and lower leakage currents, making it preferable for modern electronic applications \cite{Collings1980}. These differences influence the behavior of transistors, particularly the saturation current and thermal voltage, both of which are key parameters in this experiment.

These advancements laid the foundation for modern experimental techniques. Early measurements were manual and prone to errors, while today’s experiments leverage automated voltage sweeps with precise current measurements using instruments like the Keithley picoammeter. Controlled temperature environments, achieved through heating setups and thermocouples, allow for detailed analysis of the temperature dependence of the collector current \(I_C\). These techniques facilitate the verification of theoretical models such as the Ebers-Moll equation and the determination of constants like Boltzmann's constant and the semiconductor band gap, as demonstrated in this experiment.


%=====================================================
%============ Section ================================

\section{Theoretical background}

Semiconductors are materials characterized by an electronic structure consisting of a valence band filled with electrons and a conduction band that is nearly empty at low temperatures. These bands are separated by an energy gap, known as the band gap \( V_g \). At absolute zero, semiconductors behave as insulators, with no electrons in the conduction band. However, at finite temperatures, some electrons gain enough thermal energy to cross the band gap, enabling electrical conduction. The likelihood of electron excitation across the band gap is determined by the Boltzmann factor:

\[
P(E) \propto e^{-E/kT}
\]

where \( E \) is the energy required for excitation, \( k \) is Boltzmann’s constant, and \( T \) is the absolute temperature\cite{Thornton}.

An NPN bipolar junction transistor (BJT) consists of two n-doped regions (the emitter and collector) separated by a thin p-doped base. When a positive voltage is applied to the base relative to the emitter (\( V_{BE} \)), it reduces the potential barrier, allowing electrons from the emitter to be injected into the base. Due to the thinness of the base and the applied positive voltage at the collector, most of these electrons drift into the collector, resulting in a measurable collector current \( I_C \) \cite{Neudeck}.

The relationship between the collector current \( I_C \) and the base-emitter voltage \( V_{BE} \) is described by the Ebers-Moll equation:

\[
I_C = I_S \left(e^{\frac{eV_{BE}}{kT}} - 1\right)
\]

where \( I_S \) is the saturation current, \( e \) is the elementary charge, \( k \) is Boltzmann’s constant, and \( T \) is the absolute temperature. This equation illustrates the exponential dependence of \( I_C \) on \( V_{BE} \), a fundamental property that allows transistors to function as amplifiers and switches in electronic circuits\cite{Neudeck}.

The saturation current \( I_S \) itself is temperature-dependent and follows the relation:

\[
I_S \propto e^{-V_g/kT}
\]

This indicates that \( I_S \) increases exponentially with temperature as more electrons acquire sufficient energy to cross the band gap \( V_g \). By plotting \( \ln(I_S) \) against \( 1/T \), the band gap of the semiconductor can be extracted from the slope of the linear fit:

\[
\ln(I_S) = -\frac{V_g}{kT} + \text{constant}
\]

Additionally, analyzing the exponential dependence of \( I_C \) on \( V_{BE} \) at various temperatures allows for the determination of Boltzmann’s constant. The thermal voltage \( V_T \), defined as:

\[
V_T = \frac{kT}{e}
\]

also plays a crucial role in the transistor’s behavior, influencing the rate at which \( I_C \) increases with \( V_{BE} \)\cite{Collings1980}.

In this experiment, we measure \( I_C \) as a function of \( V_{BE} \) at different temperatures to extract both Boltzmann’s constant and the semiconductor band gap. These measurements validate theoretical models and provide insight into the intrinsic properties of semiconductor materials like silicon and germanium.


%=====================================================
%============ Section ================================

\section{Experimental setup}


\subsection{Apparatus}

Ideas behind the particular technique should be briefly
discussed. Enclose references. Sketches, pictures, and
suitable schematics should be included and explained
concisely. All major components of the system should be
mentioned and their role clearly motivated. This section
is not simply a list of components and it is not an
instruction manual. 


%=====================================================
%============ Importing pictures  ====================
%=====================================================

\begin{figure}[h]
  %
  % placement specifier = { h,t,b,p,!,H }
  % see the following url for placement specifier definitions:
  % http://en.wikibooks.org/wiki/LaTeX/Floats,_Figures_and_Captions
  %
 \begin{center}
 \includegraphics*[width=3.5in]{string_theory.pdf}
  %
 \caption{ My Caption, in all its glory.\label{fig:apparatus} }
 % See http://en.wikibooks.org/wiki/LaTeX/Labels_and_Cross-referencing
 %  for information on labels.
 \end{center}
\end{figure}

\subsection{Data Collection}

Data taking procedures should be described and various modes of
data collection explained. Calibration procedures and
relevant plots and numerical tables should be included.
State clearly what measurements were taken for the final
data analysis. Describe `doing the experiment' so it would
be helpful to other students in the future. This may need
to include physics arguments {\em what } and {\em how } data should
be collected.


\subsection{Data Analysis}

This is the most important section of the report.
Describe data analysis. Details! Perhaps include a figure and refer
the reader to it! See Figure~\ref{fig:apparatus}. Maybe you will need
to include a table. See Table~\ref{tab:events}.

Describe calculations of the final results.
Thoroughly address error analysis and discussion of measurement
uncertainties. Remember: NO EXPERIMENTAL RESULT CAN BE QUOTED
WITHOUT AN ERROR BAR! Do not forget about random or systematic
uncertainties. Be sure to propagate errors correctly!
Include a demonstrative graph when possible.
%See Figure~\ref{fig:results}.


Make final assessment and interpretation after that.
Discuss apparatus problems if any. Suggestions for
lab setup or approach improvements are welcome!

%=====================================================
%============ Importing pictures  ====================
%=====================================================

\begin{figure}[h]
  %
  % placement specifier = { h,t,b,p,!,H }
  % see the following url for placement specifier definitions:
  % http://en.wikibooks.org/wiki/LaTeX/Floats,_Figures_and_Captions
  %
 \begin{center}
 \includegraphics*[width=3.5in]{centrifugal_force.pdf}
  %
  %
 %\caption{ The experiment provided interesting results.\label{fig:results} }
 % See http://en.wikibooks.org/wiki/LaTeX/Labels_and_Cross-referencing
 %  for information on labels.
 \end{center}
\end{figure}


%===========================================================================
%=========================== Table 1 =======================================
%===========================================================================
%
% Note: the position of the table does not always depend on its position here. See
% http://en.wikibooks.org/wiki/LaTeX/Tables
% for details.
%

\begin {table}[h]
{
{%\footnotesize
\begin {center}
\begin {tabular} {c | c c  c | c | c c }
\hline\hline
Run 			&   ~~POT~~ 		&
\multicolumn{2}{ | c } {Predicted}  &   \multicolumn{2}  {| c} {Selected} \\
Period		& $(10^{20})$	&
\multicolumn{2}{ | c } {(No oscillations)}  &   \multicolumn{2}  {| c} {(Far Detector)} \\
			    &
			& \multicolumn{1} {| c } {~~~Fully} & \multicolumn{1} { c } {~~~Partially}
			& \multicolumn{1} {| c } {~~~Fully} & \multicolumn{1} { c } {~~~Partially} \\
			
\hline
I			& 1.269		
			& \multicolumn{1} {| r } {426 } & \multicolumn{1} { r } {375 }
		     	& \multicolumn{1} {| r } {318 } & \multicolumn{1} { r } {357 } \\

II		     	& 1.943
			& \multicolumn{1} {| r } {639 } & \multicolumn{1} { r } {565 }
		    	& \multicolumn{1} {| r } {511 } & \multicolumn{1} { r } {555 } \\

\hline
Total			& 7.246
			& \multicolumn{1} {| r } {2,451 } & \multicolumn{1} { r } {2,206 }
		     	& \multicolumn{1} {| r } {1,986 } & \multicolumn{1} { r } {2,017 } \\

\hline% \hline
\end {tabular}
\end {center}
}
}
\caption {\label{tab:events}
Predicted and observed numbers of events classified in the Far Detector as fully and
partially reconstructed charged current interactions shown for all running periods.
 }
\end {table}


%=====================================================
%============ Section ================================
%=====================================================

\section{Results}

Clearly present the result of your analysis. Make sure
you include the uncertainties. No experimental result
can be quoted without an error attached to it.

Your results should be compared with predictions and other
measurements.


%=====================================================
%============ Section ================================

\section{Summary and conlcusions}

Summarize briefly the results of the experiment.
Acknowledge (i.e., thank for) contributions or help
of your partner(s) and or
others (TA, machine shop, software used, ...).

%=====================================================
%============ Bibliography  ==========================
%=====================================================

\begin{thebibliography}{9}

\bibitem{Collings1980} 
P. J. Collings, 
\textit{Simple measurement of the band gap in silicon and germanium}, 
American Journal of Physics, \textbf{48}(3), 197--199 (1980). 
\href{https://doi.org/10.1119/1.12172}{https://doi.org/10.1119/1.12172}

\bibitem{Lukasiak} 
L. Łukasiak and A. Jakubowski, 
"History of Semiconductors," 
\textit{Journal of Telecommunications and Information Technology}, vol. 1, pp. 3–9, 2023. 
Available: \url{https://doi.org/10.26636/jtit.2010.1.1015}

\bibitem{Neudeck} 
G. W. Neudeck, 
\textit{The Bipolar Junction Transistor}, 2nd ed., 
Addison-Wesley, 1989.

\bibitem{Thornton} 
S. T. Thornton and A. F. Rex, 
\textit{Modern Physics for Scientists and Engineers}, 3rd ed., 
Thomson, Brooks/Cole, 2006.

\end{thebibliography}

%=====================================================
%============ End ====================================
%=====================================================

\end{document}

%=====================================================
%============ End ====================================
%=====================================================
