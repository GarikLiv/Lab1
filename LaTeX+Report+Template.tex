%======   Template created by Jonathan Blair  ========
%=====================================================

%=====================================================
%============ Controls ===============================
%=====================================================

\documentclass[12pt,letterpaper,onecolumn]{article}
%\documentclass[11pt,letterpaper,onecolumn]{article}
%\documentclass[10pt,letterpaper,onecolumn]{article}
%\documentclass[12pt,letterpaper,twocolumn]{article}
%\documentclass[11pt,letterpaper,twocolumn]{article}
%\documentclass[10pt,letterpaper,twocolumn]{article}


\usepackage{amsmath}
\usepackage{graphics}
\usepackage{graphicx} %more modern version of graphics
%\graphicspath{{path-to-folder-containing-necessary-graphics}{other folder as necessary}}


%=====================================================
%============ \begin{document} =======================
%=====================================================

\begin{document}

%=====================================================
%============ Title ==================================
%=====================================================

\title{Investigating the current running through a npn BJT transistor}
%\title{\Large\bf Larger, Bolded Title}

%=====================================================
%============ Author =================================
%=====================================================
\author{
 Garik Livingston and Lily Nguyen \\*
  \\*
 PHY 353L Modern Laboratory \\*
 Department of Physics \\*
 The University of Texas at Austin \\*
 Austin, TX 78712, USA
}
\date{February 11, 2025}

%\address{The University of Texas, Austin, Texas, 78712}

\maketitle

%=====================================================
%============ Abstract ===============================
%=====================================================

\begin{abstract}
	In this experiment we varied the voltage supplied to the base of a npn BJT transistor, and measured the current running through the collector to form a data set which we fitted to the Ebbers-Moll equation.
	From this fit we calculated the thermal voltage ($V_T$) and saturation current ($I_s$). Both these paramaters have strong temperature dependence and from our fit we determined an experimental value for the boltzman constant of $k = 1.378 \times 10^{-23} \pm 8.4 \times 10^{-26} \frac{J}{K}$ for the silicon transistor and a value of k = $5.6 \times 10^{-23}\pm 4.0000 \times 10^{-24} \frac{J}{K}$ for the germanium transistor. 
	
%The npn BJT transitor functions with two n doped semiconductor regions called the collector and emitter sandwiched between a p doped base region. 
\end{abstract}

%=====================================================
%============ Body of the article ====================
%=====================================================

%=====================================================
%============ Section ================================

\section{Introduction}

\subsection{Physics Motivation}

Semiconductors are essential components of modern electronics due to their unique electronic properties: a valence band filled with electrons and a conduction band that is nearly empty, separated by an energy band gap \( V_g \). The ability of electrons to transition from the valence band to the conduction band under specific conditions, such as thermal excitation or applied voltage, forms the basis for the functionality of semiconductor devices like transistors \cite{Thornton}.

In this experiment, we explore the behavior of an NPN bipolar junction transistor (BJT) and focus on the relationship between the collector current \( I_C \) and the base-emitter voltage \( V_{BE} \). The transistor’s function depends on the exponential dependence of \( I_C \) on \( V_{BE} \), described by the Ebers-Moll equation:

\[
I_C = I_S \left(e^{eV_{BE}/kT} - 1\right)
\]

where \( I_S \) is the saturation current, \( e \) is the elementary charge, \( k \) is Boltzmann’s constant, and \( T \) is the absolute temperature \cite{Neudeck}. This exponential behavior arises from the Boltzmann factor, which governs the probability of electron excitation across the band gap:

\[
P(E) \propto e^{-E/kT}
\]

By measuring \( I_C \) as a function of \( V_{BE} \) at different temperatures, we can extract both Boltzmann’s constant and the band gap of the semiconductor material. This experiment not only reinforces theoretical principles of semiconductor physics but also demonstrates the practical application of transistors in amplification and switching, which are fundamental to all modern electronic devices \cite{Thornton}.


%=====================================================
%============ Section ================================

\section{Theoretical background}

Provide some more theoretical details for your measurements.
Give formulas and references which provide a specific theoretical
context for your measurements.


%=====================================================
%============ Section ================================

\section{Experimental setup}


\subsection{Apparatus}

Ideas behind the particular technique should be briefly
discussed. Enclose references. Sketches, pictures, and
suitable schematics should be included and explained
concisely. All major components of the system should be
mentioned and their role clearly motivated. This section
is not simply a list of components and it is not an
instruction manual. 


%=====================================================
%============ Importing pictures  ====================
%=====================================================

\begin{figure}[h]
  %
  % placement specifier = { h,t,b,p,!,H }
  % see the following url for placement specifier definitions:
  % http://en.wikibooks.org/wiki/LaTeX/Floats,_Figures_and_Captions
  %
 \begin{center}
 \includegraphics*[width=3.5in]{string_theory.pdf}
  %
 \caption{ My Caption, in all its glory.\label{fig:apparatus} }
 % See http://en.wikibooks.org/wiki/LaTeX/Labels_and_Cross-referencing
 %  for information on labels.
 \end{center}
\end{figure}

\subsection{Data Collection}

Data taking procedures should be described and various modes of
data collection explained. Calibration procedures and
relevant plots and numerical tables should be included.
State clearly what measurements were taken for the final
data analysis. Describe `doing the experiment' so it would
be helpful to other students in the future. This may need
to include physics arguments {\em what } and {\em how } data should
be collected.


\subsection{Data Analysis}

This is the most important section of the report.
Describe data analysis. Details! Perhaps include a figure and refer
the reader to it! See Figure~\ref{fig:apparatus}. Maybe you will need
to include a table. See Table~\ref{tab:events}.

Describe calculations of the final results.
Thoroughly address error analysis and discussion of measurement
uncertainties. Remember: NO EXPERIMENTAL RESULT CAN BE QUOTED
WITHOUT AN ERROR BAR! Do not forget about random or systematic
uncertainties. Be sure to propagate errors correctly!
Include a demonstrative graph when possible.
%See Figure~\ref{fig:results}.


Make final assessment and interpretation after that.
Discuss apparatus problems if any. Suggestions for
lab setup or approach improvements are welcome!

%=====================================================
%============ Importing pictures  ====================
%=====================================================

\begin{figure}[h]
  %
  % placement specifier = { h,t,b,p,!,H }
  % see the following url for placement specifier definitions:
  % http://en.wikibooks.org/wiki/LaTeX/Floats,_Figures_and_Captions
  %
 \begin{center}
 \includegraphics*[width=3.5in]{centrifugal_force.pdf}
  %
  %
 %\caption{ The experiment provided interesting results.\label{fig:results} }
 % See http://en.wikibooks.org/wiki/LaTeX/Labels_and_Cross-referencing
 %  for information on labels.
 \end{center}
\end{figure}


%===========================================================================
%=========================== Table 1 =======================================
%===========================================================================
%
% Note: the position of the table does not always depend on its position here. See
% http://en.wikibooks.org/wiki/LaTeX/Tables
% for details.
%

\begin {table}[h]
{
{%\footnotesize
\begin {center}
\begin {tabular} {c | c c  c | c | c c }
\hline\hline
Run 			&   ~~POT~~ 		&
\multicolumn{2}{ | c } {Predicted}  &   \multicolumn{2}  {| c} {Selected} \\
Period		& $(10^{20})$	&
\multicolumn{2}{ | c } {(No oscillations)}  &   \multicolumn{2}  {| c} {(Far Detector)} \\
			    &
			& \multicolumn{1} {| c } {~~~Fully} & \multicolumn{1} { c } {~~~Partially}
			& \multicolumn{1} {| c } {~~~Fully} & \multicolumn{1} { c } {~~~Partially} \\
			
\hline
I			& 1.269		
			& \multicolumn{1} {| r } {426 } & \multicolumn{1} { r } {375 }
		     	& \multicolumn{1} {| r } {318 } & \multicolumn{1} { r } {357 } \\

II		     	& 1.943
			& \multicolumn{1} {| r } {639 } & \multicolumn{1} { r } {565 }
		    	& \multicolumn{1} {| r } {511 } & \multicolumn{1} { r } {555 } \\

\hline
Total			& 7.246
			& \multicolumn{1} {| r } {2,451 } & \multicolumn{1} { r } {2,206 }
		     	& \multicolumn{1} {| r } {1,986 } & \multicolumn{1} { r } {2,017 } \\

\hline% \hline
\end {tabular}
\end {center}
}
}
\caption {\label{tab:events}
Predicted and observed numbers of events classified in the Far Detector as fully and
partially reconstructed charged current interactions shown for all running periods.
 }
\end {table}


%=====================================================
%============ Section ================================
%=====================================================

\section{Results}

Clearly present the result of your analysis. Make sure
you include the uncertainties. No experimental result
can be quoted without an error attached to it.

Your results should be compared with predictions and other
measurements.


%=====================================================
%============ Section ================================

\section{Summary and conlcusions}

Summarize briefly the results of the experiment.
Acknowledge (i.e., thank for) contributions or help
of your partner(s) and or
others (TA, machine shop, software used, ...).

%=====================================================
%============ Bibliography  ==========================
%=====================================================

\begin{thebibliography}{9}

\bibitem{Lukasiak} 
L. Łukasiak and A. Jakubowski, 
"History of Semiconductors," 
\textit{Journal of Telecommunications and Information Technology}, vol. 1, pp. 3–9, 2023. 
Available: \url{https://doi.org/10.26636/jtit.2010.1.1015}

\bibitem{Neudeck} 
G. W. Neudeck, 
\textit{The Bipolar Junction Transistor}, 2nd ed., 
Addison-Wesley, 1989.

\bibitem{Thornton} 
S. T. Thornton and A. F. Rex, 
\textit{Modern Physics for Scientists and Engineers}, 3rd ed., 
Thomson, Brooks/Cole, 2006.

\end{thebibliography}

%=====================================================
%============ End ====================================
%=====================================================

\end{document}

%=====================================================
%============ End ====================================
%=====================================================
